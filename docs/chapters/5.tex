\chapter{Järjestelmän rakenne}

\section{Yleisrakenne}

Järjestelmä perustuu MVC-arkkitehtuuriin, jota sovelletaan sekä palvelimen että selaimen puolella. Osa toiminnoista on toteutettu REST-rajapintoina, joiden toteutukset löytyvät kansiosta \texttt{server/controllers/api/}. Palvelimen kaikki sivut ja rajapinnat rekisteröidään tiedostossa \texttt{server/Routes.ts}.\\

\noindent
Käyttäjille näkyvät sivut koostuvat neljästä komponentista:

\begin{itemize}
	\item[Malli] Mallit haetaan tietokannasta ja/tai luodaan kyselyn perusteella. Suurimmalle osalle tietokantatauluista on tyyppikuvaukset, jotka sijaitsevat kansiossa \texttt{server/models/}.
	\item[Näkymä] Näkymät on kirjoitettu Jade-templatekielellä. Staattinen osa näkymän sisällöstä täytetään palvelimella, ja dynaaminen sisältö täytetään Angular.js:n avulla. Templatet löytyvät kansiosta \texttt{views/}.
	\item[Palvelinkontrolleri] Palvelinkontrollerit luovat kyselyyn perustuvat mallit, hakevat tietokannasta tietueita, yhdistävät ne näkymiin ja lähettävät tuloksen käyttäjälle. Myös REST-rajapinnoille on palvelinkontrollerit, vaikka muut komponentit puuttuvat niistä. Ne löytyvät kansiosta \texttt{server/controllers/}
	\item[Käyttöliittymäkontrolleri] Angular.js:llä toteutettuja käyttöliittymäkontrollereita ajetaan käyttäjän selaimella. Ne huolehtivat dynaamisesta sisällöstä, lomakkeiden vahvistamisesta sekä rajapintakyselyistä. Ne löytyvät kansiosta \texttt{client/js/controllers/}.
\end{itemize}

\noindent
Kommunikointi tietokannan kanssa tapahtuu pääasiallisesti DAOilla eli Data Access Objecteilla, jotka löytyvät kansiosta \texttt{server/database/daos/}. Lähes jokainen niistä perustuu BaseDao-luokkaan, jossa on suurin osa tietokantalogiikasta. BaseDao löytyy tiedostosta \texttt{server/database/BaseDao.ts}.

Kirjautuminen ja sessionhallinta on toteutettu kansion \texttt{server/authentication/} tiedostoilla. Vastaavasti tunnistautuminen (eli mitä eri käyttäjät saavat tehdä) on toteutettu kansion \texttt{server/authorization/} tiedostoilla.
\newpage

\section{Tiedostojen nimeämiskäytännöt}
\begin{itemize}
	\item Palvelimen TypeScript-tiedostot käyttävät \textbf{PascalCasea}.
	\item Selaimessa ajettavat TypeScript-tiedostot käyttävät \textbf{snake\_casea}.
	\item Näkymämallit sekä tyylitiedostot käyttävät \textbf{snake\_casea}.
	
\end{itemize}
